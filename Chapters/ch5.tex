\newpage
\section[Differentiation]{\hyperlink{toc}{Differentiation}}

\subsection{Derivatives}
\begin{definition}{Derivatives}{5.1}
    Let $f: [a, b] \mapsto \RR$, and $x \in [a, b]$. We then define the \textbf{derivative} of $f$ at $x$ as:
    \begin{align*}
        f'(x) = \lim_{t\rightarrow x} \frac{f(t) - f(x)}{t - x}
    \end{align*}
    If the limit exists. Alternative notations for the derivative are given by:
    \begin{align*}
        \dpd{f}{x}(x) \text{ or } \dod{}{x}f(x) \text { or } \left.\dod{}{y}f(x)\right|_{y = x}
    \end{align*}
\end{definition}
As an interpretation of the derivative, take $[a, b]$ to be a metric space, with $x$ a limit point of $[a, b] \setminus \set{x}$. Then, $g(t) = \frac{f(t) - f(x)}{t - x}$ is a function from $[a, b] \setminus \set{x} \mapsto \RR$. If $x \in (a, b)$, then the above definition of the derivative agrees with the definition of $f'(x)$ from first year calculus. If $x = a$ or $x = b$, then the above definition agrees with the definition of the one-sided derivative from first year calculus. Note that we will not discuss in this class cases where the domain gets more complicated (i.e. not just closed intervals of $\RR$).

\begin{theorem}{}{5.2}
    Let $f:[a,b] \mapsto \RR$, let $x \in [a, b]$, and suppose $f'(x)$ exists. Then, $f$ is continuous at $x$.
\end{theorem}
\begin{nproof}
    For $t \neq x$, we can write:
    \begin{align*}
        f(t) = f(x) + (f(t) - f(x)) = f(x) + \frac{f(t) - f(x)}{t - x}(t - x)
    \end{align*}
    Taking the limit of $t \rightarrow x$, we then have that:
    \begin{align*}
        \lim_{t \rightarrow x} f(t) = \lim_{t \rightarrow x} \left(f(x) + \frac{f(t) - f(x)}{t - x}(t - x) \right) = \lim_{t \rightarrow x} f(x) + \lim_{t \rightarrow x} \frac{f(t) - f(x)}{t - x} \lim_{t \rightarrow x} (t - x)
    \end{align*}
    Where in the last line we invoke Theorem \ref{thm:4.4}. Evaluating the limits on the RHS by using the existence of the derivative of $f$ at $x$, we have
    \begin{align*}
        \lim_{t \rightarrow x} f(t) = f(x) + f'(x)\cdot (0) = f(x)
    \end{align*}
    So we conclude that $f$ is continuous at $x$ by Theorem \ref{thm:4.6}. \qed
\end{nproof}
The interpretation is that differentiability at $x \in (a, b)$ implies continuity of $f$ at $x$, and the left/right differentiability of $f$ at $a/b$ implies the left/right continuity of $f$ at $a/b$. We have wrapped the proof of all these cases into one!

Note that the converse of the above theorem is not true. As a simple example, take $f(x) = \abs{x}$ on $[-1, 1]$, which is continuous at $x = 0$ (it can be verified that $\lim_{x \rightarrow 0}f(x) = f(0) = 0$) but is not differentiable there (the left/right handed limits of the difference quotient do not agree and hence the derivative does not exist). In Chapter 7, we will construct a function that is continuous everywhere and differentiable nowhere!

NWe will now proceed to prove a series of theorems that have been seen in first year, but using our new/rigorous definitions.

\begin{theorem}{Sum, Product, and Quotient Rules}{5.3}
    Let $f, g: [a, b] \mapsto \RR$. Let $x \in [a, b]$ and suppose $f$ and $g$ are differentiable at $x$. Then, $f + g$, $f - g$, $f\cdot g$ are differentiable at $x$, and so is $\frac{f}{g}$ provided $g(x) \neq 0$. Furthermore:
    \begin{enumerate}
        \item $(f+g)'(x) = f'(x) + g'(x)$
        \item $(fg)'(x) = f'(x)g(x) + f(x)g'(x)$
        \item $\left(\frac{f}{g}\right)'(x) = \frac{f'(x)g(x) - f(x)g'(x)}{(g(x))^2}$
    \end{enumerate}
\end{theorem}
\begin{nproof}
    \begin{enumerate}
        \item Follows immediately from the additive property of limits (Theorem \ref{thm:4.4}).
        \item Let $h = fg$. We then have that:
        \begin{align*}
            h(t) - h(x) = f(t)\left[g(t) - g(x)\right] + g(x)\left[f(t) - f(x)\right]
        \end{align*}
        For $t \neq x$, we can divide both sides by $t-x$ to obtain:
        \begin{align*}
            \frac{h(t) - h(x)}{t - x} = f(t)\frac{g(t) - g(x)}{t-x} + g(x)\frac{f(t) - f(x)}{t - x}
        \end{align*}
        Taking the limit of $t \rightarrow x$ on both sides, we obtain:
        \begin{align*}
            h'(x) = f(x)g'(x) + f'(x)g(x)
        \end{align*}
        as desired.
        \item Let $h(t) = \frac{f(t)}{g(t)}$. Then:
        \begin{align*}
            h(t) - h(x) &= \frac{f(t)}{g(t)} - \frac{f(x)}{g(x)}
            \\ &= \frac{1}{g(t)g(x)}\left(f(t)g(x) - g(t)f(x)\right)
            \\ &= \frac{1}{g(t)g(x)}\left[g(x)\left(f(t) - f(x)\right) - f(x)\left(g(t) - g(x)\right)\right]
        \end{align*}
        For $t \neq x$, we can divide both sides by $t - x$ to get:
        \begin{align*}
            \frac{h(t) - h(x)}{t - x} = \frac{1}{g(t)g(x)}\left[g(t)\frac{f(t) - f(x)}{t - x} - f(x)\frac{g(t) - g(x)}{t - x}\right]
        \end{align*}
        Taking the limit as $t \rightarrow x$ on both sides, we obtain the desired expression. \qed
    \end{enumerate}
\end{nproof}

\newpage 
\noindent As an exercise, one can prove by induction (applying 5.3(b)) that $(f_1f_2f_3\ldots f_n)'(x)$ (where $f_i: [a, b] \mapsto \RR$ and each $f_i'(x)$ exists) is given by:
\begin{align*}
    f_1'(x)f_2(x)\ldots f_n(x) + \cdots + f_1(x)f_2(x)\ldots f'_n(x).
\end{align*}
Note that as a corollary of this, we get that if $f(x) = x^n$, then $f'(x) = nx^{n-1}$ and we hence recover the familiar power rule from first year calculus!

\setcounter{rudin}{4}
\begin{theorem}{Chain Rule}{5.5}
    Let $f: [a, b] \mapsto \RR$, $x \in [a, b]$, and suppose $f$ is differentiable at $x$. Suppose furthermore that $f([a, b])$ is contained in some interval $I$. Let $g: I \mapsto \RR$ and suppose $g$ is differentiable at $f(x)$. Then, $g \circ f: [a, b] \mapsto \RR$ is differentiable at $x$, and furthermore:
    \begin{align*}
        (g \circ f)'(x) = g'(f(x))f'(x)
    \end{align*}
\end{theorem}
\begin{nproof}
    Define $h(t) = g \circ f(t)$ for $a \leq t \leq b$, $t \neq x$. We cna then write:
    \begin{align*}
        f(t) - f(x) = (t-x)\left[f'(x) + u(t)\right]
    \end{align*}
    For a function $u(t)$ with $\lim_{t \rightarrow x} u(t) = 0$. Now defining $y = f(x)$, we write:
    \begin{align*}
        g(s) - g(y) = (s-y)\left[g'(y) + r(s)\right]
    \end{align*}
    For a function $r(s)$ with $\lim_{s \rightarrow y}r(s) = 0$. Hence, we have that:
    \begin{align*}
        h(t) - h(x) &= g(f(t)) - g(f(x))
        \\ &= \left(f(t) - f(x)\right)\left(g'(y) + r(s)\right)
        \\ &= (t- x)\left[f'(x) + u(t)\right]\left(g'(y) + r(s)\right)
    \end{align*}
    Dividing both sides by $t - x$, we obtain:
    \begin{align*}
        \frac{h(t) - h(x)}{t - x} = \left[f'(x) + u(t)\right]\left(g'(y) + r(s)\right)
    \end{align*}
    We now take the limit of $t \rightarrow x$ on both sides. $\lim_{t \rightarrow x} u(t) = 0$, and $f$ is differentiable and hence continuous at $x$, so $s = f(t) \rightarrow y$ as $t \rightarrow x$. Thus, $r(s) \rightarrow 0$ as $t \rightarrow x$, and in conclusion:
    \begin{align*}
        h'(x) = (g \circ f)'(x) =  f'(x)g'(y) = g'(f(x))f'(x)
    \end{align*}
    as desired. \qed
\end{nproof}


\subsection{MVT}

\setcounter{rudin}{6}
\begin{definition}{Local Maxima/Minima}{5.7}
    Let $X$ be a metric space. Let $f: X \mapsto \RR$, and let $x \in X$. We say that $x$ is a \textbf{local maximum} of $f$ if there exists $\delta > 0$ such that $f(y) \leq f(x)$ for all $y \in N_{\delta}(x)$. A \textbf{local minimum} is defined similarly, with $f(y) \geq f(x)$ instead.
\end{definition}
\noindent For a metric space $X$ equipped with the discrete metric, all points $x \in X$ are simultaneously local maxima and minima. To see this, take any $0 < \delta \leq 1$. 

\begin{theorem}{}{5.8}
    Let $f: [a, b] \mapsto \RR$. Let $x \in [a, b]$ and suppose that $f'(x)$ exists, and $f$ is either a local maximum or local minimum of $f$. Then, $f'(x) = 0$. 
\end{theorem}
\begin{nproof}
    Suppose $x$ is a local minimum. Then, there exists $\delta > 0$ such that $N_{\delta}(x) \subset [a, b]$, and $f(y) \geq f(x)$ for all $y \in N_{\delta}(x)$. Thus, if $x < y < x + \delta$, then:
    \begin{align*}
        \frac{f(y) - f(x)}{y - x} \geq 0 \implies f'(x) \geq 0
    \end{align*}
    Conversely, if $x - \delta < y < x$, then:
    \begin{align*}
        \frac{f(y) - f(x)}{y - x} \leq 0 \implies f'(x) \leq 0
    \end{align*}
    So taken together we obtain that $f'(x) = 0$. An identical argument is used for the case of a local maximum. \qed
\end{nproof}

\begin{ntheorem}{: Rolle's Theorem}{}
    Let $f: [a, b] \mapsto \RR$ be continuous, and suppose $f$ is differentiable on $(a, b)$. If $f(a) = f(b)$, then there exists $x \in (a, b)$ such that $f'(x) = 0$.
\end{ntheorem}
\begin{nproof}
    Since $[a, b]$ is compact and $f$ is continuous, by the EVT (Theorem \ref{thm:4.16}) $f$ attains its maximum on $[a, b]$, that is, there exists $c \in [a, b]$ such that $f(y) \leq f(x)$ for all $y \in [a, b]$. If $c \in (a, b)$, then by Theorem \ref{thm:5.8}, $f'(c) = 0$ and we are done. Next, suppose $c = a$ or $c = b$. Again by the EVT, $f$ attains its minumum on $[a, b]$, that is, there exists $d \in [a, b]$ such that $f(y) \geq f(d)$ for all $y \in [a, b]$. If $d \in (a, b)$, then by Theorem \ref{thm:5.8}, $f'(d) = 0$ and we are done. Suppose then that $d = a$ or $d = b$. Since $f(a) = f(b)$, we therefore obtain that $f(a) = f(b) = f(c) = f(d)$ and the maximum/minimum values agree. Hence, $f(y) = f(a)$ for all $y \in [a, b]$, so $f'(y) = 0$ for all $y \in [a, b]$. So, the desired $x$ may be any point in $[a, b]$. \qed
\end{nproof}

\begin{figure}[htbp]
    \centering
    \begin{tikzpicture}[scale = 2]
        \draw[-latex] (0, 0) -- (0, 2);
        \draw[-latex] (0, 0) -- (2, 0);
        \draw[] (1, 1) parabola (0.5, 0.2);
        \draw[] (1, 1) parabola (1.5, 0.2);
        \filldraw[] (0.5, 0.2) circle (1pt);
        \filldraw[] (1.5, 0.2) circle (1pt);
        \draw[dashed] (0, 1.01) -- (2, 1.01);
        \draw[] (0.5, 0) -- (0.5, -0.15);
        \node[below] at (0.5, -0.15) {$a$};
        \draw[] (1.5, 0) -- (1.5, -0.15);
        \node[below] at (1.5, -0.12) {$b$};
        \draw[] (1, 0) -- (1, -0.15);
        \node[below] at (1, -0.16) {$x$};
    \end{tikzpicture}
    
    \caption{A simple parabolic function that demonstrates Rolle's Theorem.}
    \label{fig21}
\end{figure}

\setcounter{rudin}{9}
\begin{theorem}{Mean Value Theorem}{5.10}
    Let $f: [a, b] \mapsto \RR$ be continuosu and differentiable on $(a, b)$. Then, there exists $x \in (a, b)$ such that $f(b) - f(a) = f'(x)(b - a)$.
\end{theorem}
\noindent The visual interpretation of this theorem is that there exists $x \in (a, b)$ such that the slope of the tangent line to $f$ at $x$ is equal to the secant line slope between $(a, f(a))$ and $(b, f(b))$. The idea of the proof is to rotate one's head such that the sectant line is horizontal; one is then able to apply Rolle's Theorem!

\begin{nproof}
    Define $h(y) = f(y) = \frac{f(b) - f(a)}{b - a}(y - a)$. $h$ is continuous on $[a, b]$ and differentiable on $(a, b)$ (being a sum of continuous/differentiable functions). We have that $h(a) = f(a) - 0 = f(a)$, and $h(b) = f(b) - \frac{f(b) - f(a)}{b - a}(b - a) = f(a)$. Applying Rolle's Theorem to $h$, there exists $x \in (a, b)$ such that $h'(x) = 0$. Therefore, $h'(x) = 0 = f'(x) - \frac{f(a) - f(b)}{b - a} = 0$, and we conclude that $f(b) - f(a) = f'(x)(b - a)$ for some $x \in (a, b)$. \qed
\end{nproof}

\begin{figure}[htbp]
    \centering
    
    \caption{Figure demonstrating the MVT}
    \label{fig22}
\end{figure}


\begin{theorem}{}{5.11}
    Let $f: [a, b] \mapsto \RR$ be differentiable on $(a, b)$. Then:
    \begin{enumerate}
        \item If $f'(x) \geq 0$ for all $x \in (a, b)$, then $f$ is monotonically increasing.
        \item If $f'(x) = 0$ for all $x \in (a, b)$, then $f$ is constant. 
        \item If $f'(x) \leq 0$ for all $x \in (a, b)$, then $f$ is monotonically decreasing.
    \end{enumerate} 
\end{theorem}
\begin{nproof}
    If $a < x < y < b$, by the mean value theorem, there exists $z \in (x, y)$ such that:
    \begin{align*}
        f(y) - f(x) = f'(z)(y - x)
    \end{align*}
    Note that $y - x > 0$ by construction. 
    \begin{enumerate}
        \item If $f'(x) \geq 0$ for all $x \in (a, b)$, then $f'(z) \geq 0$, showing that $f(y) - f(x) \geq 0$ and hence that $f$ is monotonically increasing. 
        \item If $f'(x) = 0$ for all $x \in (a, b)$, then $f'(z) = 0$, showing that $f(y) - f(x) = 0$ and hence that $f$ is constant on $(a, b)$.
        \item If $f'(x) \leq 0$ for all $x \in (a, b)$, then $f'(z) \leq 0$, showing that $f(y) - f(x) \leq 0$ and hence that $f$ is monotonically decreasing. \qed
    \end{enumerate}
\end{nproof}

\subsection{Taylor's Theorem}

\subsection{Local Behavior of Functions}