\section{Basic Topology}
\subsection{Finite and Countable Sets}
This chapter is split into two portions; the first looks at counting, what it means for us to say that two sets have the same number of elements, and concludes with a classic theorem of Cantor concerning uncountable sets. The second part looks at the topology of metric spaces, before moving onto the topology of the real numbers. 

Let us then begin with our discussion of counting. If we consider counting how many bananas there are on a table (say there are 10 bananas), then what we are formally doing is establishing a correspondence between each ball on the table with an element in the set $\set{1, \ldots, 10}$. When we refer to the number of elements in a set, it will be good to keep in mind that we are establishing functions between sets. Although we have been discussing functions with some frequency in the course already, we give a definition below for completeness. 

\begin{definition}{Functions}{2.1}
    Let $A, B$ be sets. Then, a map that associates each element $x \in A$ with a unique element denoted as $f(x) \in B$ is a \textbf{function} $f: A \rightarrow B$. We then define $A$ as the domain of $f$ and the set $\set{f(x): x \in A}$ as the range.
\end{definition}

\subsection{Uncountable Sets}
\subsection{Topology of Metric Spaces}
\subsection{Closure and Relative Topology}
\subsection{Compactness}
\subsection{Compactness in \texorpdfstring{$\RR^k$}{TEXT} and the Cantor Set}
