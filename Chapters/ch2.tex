\newpage
\section{Basic Topology}
\subsection{Finite and Countable Sets}
This chapter is split into two portions; the first looks at counting, what it means for us to say that two sets have the same number of elements, and concludes with a classic theorem of Cantor concerning uncountable sets. The second part looks at the topology of metric spaces, before moving onto the topology of the real numbers. 

Let us then begin with our discussion of counting. If we consider counting how many bananas there are on a table (say there are 10 bananas), then what we are formally doing is establishing a correspondence between each ball on the table with an element in the set $\set{1, \ldots, 10}$. When we refer to the number of elements in a set, it will be good to keep in mind that we are establishing functions between sets. Although we have been discussing functions with some frequency in the course already, we give a definition below for completeness. 

\begin{definition}{Functions}{2.1}
    Let $A, B$ be sets. Then, a map that associates each element $x \in A$ with a unique element denoted as $f(x) \in B$ is a \textbf{function} $f: A \rightarrow B$. We then define $A$ as the \textbf{domain} of $f$ and the set $\set{f(x): x \in A}$ as the \textbf{range}. For $E \subseteq A$, we call $f(E) = \set{f(x): x \in E}$ the \textbf{image} of $E$ under $f$. For $F \subseteq B$, we call $f^{-1}(B) = \set{x \in A: f(x) \in B}$ the \textbf{preimage} of $F$. 
\end{definition}

\begin{definition}{Injective/Surjective Functions}{2.2}
    Let $f: A \mapsto B$ be a function. If for $x_1, x_2 \in A$ we have that $f(x_1) = f(x_2) \implies x_1 = x_2$ (or equivalently, $x_1 \neq x_2 \implies f(x_1) \neq f(x_2)$), then we say that $f$ is \textbf{injective}, or \textbf{one-to-one}. If for all $y \in B$ there exists $x \in A$ such that $y = f(x)$, then we say that $f$ is \textbf{surjective}, or \textbf{onto}. If a function is both injective and surjective, it is \textbf{bijective}.
\end{definition}
\noindent Intuitively, we can think of injectivity as implying each element in $B$ being reached at most once, and surjectivity implying that each element in $B$ is reached at least once. 

\begin{definition}{Cardinality \& Equivalence}{2.3}
    Let $A, B$ be sets. We say that $A, B$ have the same \textbf{cardinality} if there exists $f: A \mapsto B$ such that $f$ is bijective. We can denote this as $A \sim B$ where $\sim$ indicates an \textbf{equivalence relation}. An equivalence relation has three properties:
    \begin{enumerate}
        \item Reflexivity: $A \sim A$.
        \item Symmetry: If $A \sim B$ then $B \sim A$.
        \item Transitivity: If $A \sim B$ and $B \sim C$ then $A \sim C$.
    \end{enumerate}
    As a point of notation, $\abs{S}$ denotes the cardinality of the set $S$. 
\end{definition}

\noindent We get (a) from each set having a bijection to itself (i.e. the identity function), (b) from the fact that if there exists a bijection $f: A \mapsto B$, then there must exist an inverse $f^{-1}: B \mapsto A$, and (c) from if there exist bijections $f: A \mapsto B$ and $g: B \mapsto C$ then the composition $g \circ f: A \mapsto C$ will also be a bijection. 

\begin{definition}{Countability}{2.4}
    First, we denote $J_n = \set{1, 2, 3, \ldots, n}$ and $J = \NN = \set{1, 2, 3, \ldots}$. Let $A$ be a set. We say that $A$ is \textbf{finite} if it has a finite number of elements, that is, there exists $n \in \NN$ such that $A \sim J_n$. A set $A$ is \textbf{infinite} if it is not finite, and we cannot put $A$ in bijection with $J_n$ for any $n \in \NN$. We say that $A$ is \textbf{countable}if $A \sim \NN$, and \textbf{uncountable} otherwise. 
\end{definition}

\noindent Note that the above definition gives us a useful notion for what sets we can consider countable; if we can enumerate a set with the naturals, this yields a bijection with $\NN$ and hence the set must be countable. 

We here give some additional properties concerning cardinalities of sets, which may be useful:
\begin{itemize}
    \item $\abs{A} \leq \abs{B} \iff \exists f: A \mapsto B \text{ such that $f$ is injective}$
    \item $\abs{A} \geq \abs{B} \iff \exists f: A \mapsto B \text{ such that $f$ is surjective}$
    \item $\abs{A} \leq \abs{B} \text{ and } \abs{A} \geq \abs{B} \implies \abs{A} = \abs{B}$
\end{itemize}

\begin{example}{}{2.5}
    $\ZZ$ is countable. To see this, consider the function:
    \begin{align*}
        f = \begin{cases}
            \frac{n}{2} & \text{$n$ is even}
            \\ -\frac{n-1}{2} & \text{$n$ is odd}
            \end{cases}
    \end{align*}
    $f$ is a bijection (check!) and hence $\NN \sim \ZZ$. 
\end{example}
\noindent The above example serves as a bit of a warning sign. Even though $\NN \subsetneq \ZZ$ and $\ZZ$ has ``more elements'', we still find that the two sets have the same cardinality. A similar example is given by $\NN$ and the set of all even natural numbers (which we may denote $2\NN$); the bijection $f(n): n \mapsto 2n$ between these two sets shows that $\NN \sim 2\NN$, even though $2\NN$ is a strict subset of $\NN$. 
\setcounter{rudin}{7}

\begin{theorem}{}{2.8}
    A subset of a countable set is either finite or countable.
\end{theorem}
\begin{nproof}
    (Sketch) The countability of $A$ implies that $A = \set{a_1, a_2, a_3, a_4, a_5 \ldots}$ (in other words, we can enumerate the elements using $\NN$). Let $S \subseteq A$. Then, $S = \set{a_1, \cancel{a_2}, a_3, a_4, \cancel{a_5}, \ldots}$, that is, $A$ with some (or none) of the elements removed. Now, we can rename all the elements with $a_1, a_2, \ldots$; what we have left is again an enumeration, so it is yet again (at most) countable.
\end{nproof}
\noindent One potentially useful fact is that if we have a set $S$ and a function $f: \NN \mapsto S$ such that $f$ is surjective, then $S$ is at most countable. 

\begin{proof}
    Let $T = \set{n \in \NN: f(n) \neq f(m), \forall m = 1, 2, \ldots, m}$. We restrict $f: T \mapsto S$, then $f$ is injective by constructive. It is still surjective, hence $T \sim S$. Since $T \subset \NN$, by Theorem \ref{thm:2.8}, $S$ is finite or countable.
\end{proof}

\setcounter{rudin}{11}

\begin{theorem}{}{2.12}
    Let $E_1, E_2, \ldots$ be countable sets (i.e. we have a countable number of countable sets). Define $S = \bigcup_{n=1}^{\infty} E_n$. Then, $S$ is countable. 
\end{theorem}
\begin{nproof}
    Write $E_n = \set{x_{n1}, x_{n2}, x_{n3}, \ldots}$ (which we can do as each of the $E_n$s are countable). Then, we form an array:
    \begin{center}
    \begin{tikzpicture}
        \matrix[matrix of math nodes,inner sep=1pt,row sep=1em,column sep=1em] (M)
        {
            E_1 & = & x_{11} & x_{12} & x_{13}  & \cdots \\
            E_2 & = & x_{21} & x_{22} & x_{23}  & \cdots \\
            E_3 & = & x_{31} & x_{32} & x_{33}  & \cdots \\
            \cdots \\
        }
        ;
        \draw[->] (M-1-3.south west) -- (M-1-3.north east);
        \draw[->] (M-2-3.south west) -- (M-1-4.north east);
        \draw[->] (M-3-3.south west) -- (M-1-5.north east);
        \draw[->] (M-1-3.south west) -- (M-1-3.north east);
        \draw[->] (M-3-4.south west) -- (M-2-5.north east);
        \draw[->] (M-3-5.south west) -- (M-3-5.north east);
    \end{tikzpicture}
    \end{center}
    Then, we can re-number the elements along the diagonal lines (i.e. $x_{11}, x_{21}, x_{12}, x_{31}, x_{22}, x_{13}, \ldots$). This new enumeration corresponds to a countable set. From there, we let $T \subseteq \NN$ be the remaining labels in the enumeration after removing the repeated elements from the sequence. Then, $T \sim S$, and hence $S$ is at most countable. $S$ cannot be finite as $E_1 \subseteq S$ and $E_1$ is not finite. Hence $S$ is countable. \qed
\end{nproof}

\begin{corollary}{}{2.13}
    \begin{itemize}
        \item If $A$ is countable, the set of n-tuples of $(a_1, \ldots a_n)$ is also countable for any $n \in \NN$.  
        \item $\QQ$ is countable.
    \end{itemize}
\end{corollary}
\noindent We defined $\QQ$ as pairs of integers, but by the first part of the corollary (which follows immediately by application of Theorem \ref{thm:2.12}) $\ZZ^2$ (the set of pairs of integers) has equal cardinality to $\ZZ$, and since $\QQ$ is a subset of the set of pairs of integers, $\QQ$ is countable. 

From the discussion of today, we have established that $\abs{\NN} = \abs{\ZZ} = \abs{\QQ}$. Does $\RR$ also have equal cardinality to these sets? Do infinite sets in general have the same cardinality? The answer turns out to be no for both of these questions. We will answer the first question in the next lecture (when we discuss Cantor diagonalization, a highlight of the course), but we can discuss the second statement now. First, we make a definition:
\begin{ndef}{Power Sets}
    Let $A$ be a set. Then, the power set of $A$, denoted $\mathcal{P}(A)$, is the set of all subsets of $A$. 
\end{ndef}
\noindent An interesting theorem then follows:
\begin{ntheorem}{Power Set Cardinality}
    Let $A$ be a set. Then, $\abs{A} < \abs{\mathcal{P}(A)}$.
\end{ntheorem}
\begin{nproof}
    Suppose for the sake of contradiction that there exists a surjection $f: A\rightarrow \pset{A}$ (this would imply that $\abs{A} \geq \abs{\pset{A}}$, so by showing this is false, we obtain the desired result). Then, each element $x \in A$ gets mapped to some subset of $A$. We either have that $x$ belongs to the subset that it gets mapped to, or it doesn't. Therefore, we can define a new subset $B \subseteq A$, such that:
    \[B = \set{x\in A: x \notin f(x)}\]
    In other words, the set of all $x$s that are not in the subset that they get mapped to by $f$. Since $f$ is surjective, there must be an element $y \in A$ such that $f(y) = B$. One of $y \in B$, $y \notin B$ must be true. If $y \in B$, by construction of $B$ we have that $y$ is not in the subset that it gets mapped to by $f$, which is a contradiction. If $y \notin B$, by definition of $B$, $y \in B$ as it is not in the subset that it gets mapped to, yet again a contradiction. Therefore, we conclude that no $y \in A$ exists such that $f(y) = B$, and hence, no surjective $f$ exists such that $f: A \rightarrow \pset{A}$. Hence, $\abs{A} < \abs{\pset{A}}$. \qed
\end{nproof}
\noindent An interesting consequence of this theorem is that for a countable set $A$, we then have that $\mathcal{P}(A)$ is an infinite set which has greater cardinality! For example, $\abs{\NN} < \abs{\mathcal{P}(\NN)}$. Moreover, this gives rise to an infinite number of cardinalities in ascending order; $\abs{\NN} < \abs{\pset{\NN}} < \abs{\pset{\pset{\NN}}} < \cdots$ and so on.

\subsection{Uncountable Sets}
\begin{theorem}{}{2.14}
Let $A = \set{(b_1, b_2, \ldots), b_n \in \set{0, 1}}$ be the set of binary sequences. Then, $A$ is uncountable. 
\end{theorem}
\begin{nproof}
    It suffices to show that every countable subset of $A$ is a proper subset of $A$. Let $E \subset A$ be countable, and let $E = \set{S^{(1)}, S^{(2)}, S^{(3)}, \ldots}$. To show that $E$ is a proper subset, we show that there exists a sequence $S \in A \setminus E$. To construct such an $S$, let us put the elements of $E$ in an array. 
    \begin{center}
        \begin{tikzpicture}
            \matrix[matrix of math nodes,inner sep=1pt,row sep=1em,column sep=1em] (M)
            {
                S^{(1)} & = & \boxed{b_{1}^{1}} & b_{2}^{1} & b_{3}^{1} &  \cdots \\
                S^{(2)} & = & b_{1}^{2} & \boxed{b_{2}^{2}} & b_{3}^{2} & \cdots \\
                S^{(3)} & = & b_{1}^{3} & b_{2}^{3} & \boxed{b_{3}^{3}}  & \cdots \\
            };
        \end{tikzpicture}
    \end{center}
    Then, define:
    \[\tilde{b}_n^n = \begin{cases}
        1 & \text{if $b_n^n = 0$}
        \\ 0 & \text{if $b_n^n = 1$}
    \end{cases}\]
    I.e. $\tilde{b}_n^n$ is the bit flip of $b_n^n$. Then, let $S = (\tilde{b}_1^1, \tilde{b}_2^2, \tilde{b}_3^3, \ldots)$, that is, $S$ is the sequence of bit-flipped diagonal elements of the original array. By construction, $S \neq S^{(k)}$ as for any $S^{(k)} \in E$ as $S$ differs at the $k$th position. Hence, $S \notin E$ and therefore $E \subsetneq A$. \qed
\end{nproof}
\begin{ncorollary}{}{}
    $\mathcal{P}(\NN)$ (the power set of $\NN$) is uncountable. $\RR$ is uncountable.
\end{ncorollary}
\begin{nproof}
    Although we showed that $\mathcal{P}(\NN)$ was uncountable last lecture, we show this in an alternative way by considering a bijection between $\mathcal{P}(\NN)$ and the set $A$ of binary sequences. To do this, consider that we can associate a subset $T \subset \NN$, $T \in \mathcal{P}(\NN)$ with the sequence corresponding to:
    \begin{align*}
        b_n = \begin{cases}
            1 & \text{if $n \in T$}
            \\ 0 & \text{if $n \notin T$}
        \end{cases}
    \end{align*}
    Since $A$ is uncountable, it follows that $\mathcal{P}(\NN)$ is uncountable. The second statement in the corollary follows (roughly) by considering $\RR$ represented in binary, though this requires more justification than what we present here (we will show below that a subset of $\RR$ is uncountable). \qed
\end{nproof}
\begin{ntheorem}{}{}
    $[0, 1] \subset \RR$ is uncountable.
\end{ntheorem}
\begin{nproof}
    (Sketch) We construct a bijection from $[0, 1]$ to $A$. Let $x \in [0,1]$, and let $b_1$ be the largest integer such that $N_1 = \frac{b_1}{2} \leq x$. Then, let $b_2 \in \set{0, 1}$ be the largest integer such that $N_2 = \frac{b_1}{2} + \frac{b_2}{2^2} \leq x$. We can continue dividing $[0,1]$ in half in this way, approximating $x$ by powers of 2 (a decimal expansion in binary). Then, let $E(x) = \set{N_1, N_2, N_3, \ldots}$. By construction, $E(x)$ is bounded above by $x$ and nonempty. Hence, $\sup(E(x))$ exists and is unique, and in fact is equal to $x$ (note that we are in essence constructing an "infinite series" where the sequence of partial sums is increasing and bounded, approaching $x$ from the left). Doing this we can associate $(b_1, b_2, b_3, \ldots) \in A$ with every number $x \in [0, 1]$ and therefore $[0, 1] \sim A$. Hence $[0, 1]$ is uncountable. \qed
\end{nproof}

\subsection{Topology of Metric Spaces}
In our investigation of topology, we will try to better understand distances between and neighbourhoods of points. Distance is a familiar notion that we already have, and we will formalize this notion in the context of a metric space.

\subsection{Closure and Relative Topology}
\subsection{Compactness}
\subsection{Compactness in \texorpdfstring{$\RR^k$}{TEXT} and the Cantor Set}
