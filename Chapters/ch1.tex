\section{The Real and Complex Number Systems}
\subsection{The Naturals, Integers, and Rationals}
We begin by a review of number systems which are already familiar.

The \textbf{Naturals}, denoted by $\NN$, is the set $\set{1, 2, 3, \ldots}$. For $x, y, \in \NN$, we have that $x + y \in \NN$ and $xy \in \NN$, so the naturals are closed under addition and multiplication. However, we note that it is not closed under subtraction; take for example $2 - 4 = -2 \notin \NN$. 

The \textbf{Integers}, denoted by $\ZZ$, is the set $\set{\ldots, -3, -2, -1, 0, 1, 2, 3, \ldots}$. It is closed under addition, multiplication, and subtraction. However, it is not closed under division; for example, $1/2 \notin \ZZ$. 

The \textbf{Rationals}, denoted by $\QQ$, does not have as obvious of a denumeration. A first attempt would be $\set{\frac{m}{n}: m \in \ZZ, n \in \NN}$, where $\frac{m_1}{n_1}$ and $\frac{m_2}{n_2}$ are identified if $m_1n_2 = m_2n_1$. This is a good definition if we already have the same rigorous idea of what a rational number is in our mind; i.e. it works because we have a shared preconceived understanding of a rational number.

If this is not the case, it may help to define the rational numbers more rigorously/formally (even if the definition may be slightly harder to parse). As a second attempt at a definition, we can say that $\QQ$ is the set of ordered pairs $\set{(m, n): m \in \ZZ, n \in \NN}$. However, this is not quite enough as we need a notion of equivalence between two rational numbers (e.g. $(1, 2) = (2, 4)$). Hence, a complete and rigorous definition would be $\QQ = \set{(m, n): m \in \ZZ, n \in \NN}/\sim$ where $(m_1, n_1) \sim (m_2, n_2)$ if $m_1n_2 = m_2n_1$. 
