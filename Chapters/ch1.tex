\section{The Real and Complex Number Systems}
\subsection{The Naturals, Integers, and Rationals}
We begin by a review of number systems which are already familiar.

\begin{ndef}{The Natural Numbers}
    The \textbf{Naturals}, denoted by $\NN$, is the set $\set{1, 2, 3, \ldots}$.
\end{ndef}

For $x, y, \in \NN$, we have that $x + y \in \NN$ and $xy \in \NN$, so the naturals are closed under addition and multiplication. However, we note that it is not closed under subtraction; take for example $2 - 4 = -2 \notin \NN$.

\begin{ndef}{The Integers}
    The \textbf{Integers}, denoted by $\ZZ$, is the set $\set{\ldots, -3, -2, -1, 0, 1, 2, 3, \ldots}$.
\end{ndef}

The integers are closed under addition, multiplication, and subtraction. However, it is not closed under division; for example, $1/2 \notin \ZZ$. 

\begin{ndef}{The Rationals (informal)}
    The \textbf{Rationals}, denoted by $\QQ$, can be defined as $\set{\frac{m}{n}: m \in \ZZ, n \in \NN}$, where $\frac{m_1}{n_1}$ and $\frac{m_2}{n_2}$ are identified if $m_1n_2 = m_2n_1$.
\end{ndef}
We note that unlike the naturals/integers, the rationals do not have as obvious of a denumeration. This above is a good definition if we already have the same rigorous idea of what a rational number is in our mind; i.e. it works because we have a shared preconceived understanding of a rational number.

If this is not the case, it may help to define the rational numbers more rigorously/formally (even if the definition may be slightly harder to parse). As a second attempt at a definition, we can say that $\QQ$ is the set of ordered pairs $\set{(m, n): m \in \ZZ, n \in \NN}$. However, this is not quite enough as we need a notion of equivalence between two rational numbers (e.g. $(1, 2) = (2, 4)$). Hence, a complete and rigorous definition would be:

\begin{ndef}{The Rationals (formal)}
    The \textbf{Rationals}, denoted by $\QQ$, is the set $\set{(m, n): m \in \ZZ, n \in \NN}/\sim$ where $(m_1, n_1) \sim (m_2, n_2)$ if $m_1n_2 = m_2n_1$.
\end{ndef}
Under the formal definition, the rationals are a set of equivalence classes of ordered pairs, under the equivalence relation $\sim$. We note that the rationals are closed under addition, subtraction, multiplication, and division.

A natural question then becomes if the rationals are sufficient for doing all of real analysis. Certainly, it seems as we have a number system that is closed under all our basic arithmetic operations; but is this enough? For example, are we able to take limits just using the rationals? The answer turns out to be no (they are insufficient!) and the following example will serve as one illustration of this fact. 

\begin{example}{Incompleteness of the Rationals}{1.1}
    \begin{enumerate}
        \item There exists no $p \in \QQ$ such that $p^2 = 2$. 
        \item Let $A = \set{p \in \QQ: p > 0, p^2 < 2}$, and $B = \set{p \in \QQ: p > 0, p^2 > 2}$. Then, $\forall p \in A, \exists q \in A$ such that $p < q$, and $\forall p \in B, \exists q \in B$ such that $q < p$. 
    \end{enumerate}
\end{example}
For (a), we proceed via proof by contradiction. Recall in that these types of proof, we start with a certain wrong assumption, follow a correct/true line of reasoning, reach an eventual absurdity, and therefore conclude that the original assumption was mistaken. 
\begin{proof}
    Let us then suppose for the contradiction that there exists $p = \frac{m}{n}$ with $p^2 = 2$. We then have that not both $m, n$ are even, and hence at least one is odd. 
\end{proof}
Why can we conclude that not both $m, n$ are even in the above proof? This is the case as if $m, n$ we both even, then we could write $m = 2m'$, $n = 2n'$ for some $m', n'$, and then $p = \frac{m}{n} = \frac{2m'}{2n'} = \frac{m'}{n'}$ which we can continue until either the numerator or denominator is odd (Fundamental Theorem of Arithmetic). 