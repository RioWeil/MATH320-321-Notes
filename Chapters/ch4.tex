\newpage 
\section[Continuity]{\hyperlink{toc}{Continuity}}

\subsection{Limits and Continuity}
\begin{definition}{Limits}{4.1}
    Let $X, Y$ be metric spaces. Let $E \subset X$, and let $f: E \rightarrow Y$. Let $p \in X$ be a limit point of $E$. Then, we say that $\lim_{x\rightarrow p} f(x) = q$ or $f(x) \rightarrow q$ as $x \rightarrow p$ if there exists $q \in Y$ such that for all $e > 0$, there exists $\delta > 0$ such that for all $x \in E$ with $0 < d_X(x, p) < \delta$ we have that $d_Y(f(x), q) < \e$. 
\end{definition}
\noindent Note in the above definition that we do not care about $f(p)$, that is, the actual value of $f$ at $p$. In particular, if $p \notin E$, then $f(p)$ is not even necessarily defined. This distinction between the limit and the actual value of a function at a point becomes crucial later on when we want to define a derivative. Although we will discuss this in more detail in Chapter 5, the definition of a derivative of a function $g$ at a point $p \in \RR$ involves the function $f: \RR \rightarrow \RR$ such that:
\begin{align*}
    f(x) = \frac{g(x) - g(p)}{x - p}
\end{align*}
Evidently, the domain of $f$ does not contain the point $p$, but we are interested in the value of $f$ in the limit of $x \rightarrow p$ (which, if it exists, is the value of the derivative).

\begin{theorem}{}{4.2}
    
\end{theorem}

\subsection{Topological Characterization of Continuity}

\subsection{Continuity and Compactness}

\subsection{Uniform Continuiity, Connectedness, and IVT}

